% Settings for the default beamer theme
\documentclass[english, aspectratio=169]{beamer}
\usepackage[T1]{fontenc}
\usepackage[utf8]{inputenc}
\usepackage{adjustbox}
\usepackage{tabularx}
\usepackage{listings}
\usepackage{graphicx}
\usepackage{array}
\usepackage{babel}
\usepackage[ruled,vlined]{algorithm2e}
\usepackage{blkarray}
\SetAlgorithmName{Algoritmus}{algoritmus}{List of Algorithms}
\setcounter{secnumdepth}{3}
\setcounter{tocdepth}{3}

\makeatletter

\newcommand\makebeamertitle{\frame{\maketitle}}

% (ERT) argument for the TOC
\AtBeginDocument{%
	\let\origtableofcontents=\tableofcontents
	\def\tableofcontents{\@ifnextchar[{\origtableofcontents}{\gobbletableofcontents}}
	\def\gobbletableofcontents#1{\origtableofcontents}
}

% Theme settings
\usetheme{Frankfurt}
\usecolortheme{default}
\usefonttheme[onlymath]{serif}

% Template settings
\setbeamertemplate{navigation symbols}{}
\setbeamertemplate{blocks}[rounded][shadow=false]
\setbeamertemplate{title page}[default][colsep=-4bp, rounded=true, shadow=false]
\makeatother

% Custom color definitions
\definecolor{lightgrey}{gray}{0.95}
\definecolor{DarkerGreen}{RGB}{0,85,0} % Adjust the RGB values as needed

% Use the newly defined color in Beamer theme elements
\setbeamercolor{structure}{fg=DarkerGreen} % Changes basic structural elements to Darker Green
\setbeamercolor{title in head/foot}{bg=DarkerGreen} % Changes the title in header/footer to Darker Green

% Definitions for program code sections
\lstset{
	language=bash,
	basicstyle=\ttfamily\footnotesize, % Monospace font
	backgroundcolor=\color{lightgrey}, % Background color
	frame=single, % Frame around the code
	keywordstyle=\color{black}, % Keywords color
	commentstyle=\color{black}, % Comments color
	stringstyle=\color{red}, % Strings color
	showstringspaces=false, % Do not show spaces in strings
	breaklines=true, % Automatically break long lines
}

\lstset{
	language=python,
	basicstyle=\ttfamily\scriptsize, % Basic font style
	keywordstyle=\bfseries\color{blue}, % Keywords in bold and blue
	stringstyle=\color{red}, % Strings in red
	commentstyle=\color{green!50!black}, % Comments in green
	showstringspaces=false, % Do not show spaces in strings
	numbers=left, % Line numbers on the left
	numberstyle=\tiny\color{gray}, % Line number style
	stepnumber=1, % Line number step
	numbersep=5pt, % Distance of line numbers from code
	frame=single, % Frame around the code
	rulecolor=\color{black}, % Frame color
	tabsize=2, % Tab size
	breaklines=true, % Automatic line breaking
	breakatwhitespace=false, % Break lines at whitespace
	captionpos=b, % Caption position
	escapeinside={\%*}{*)}, % Escape to LaTeX
	morekeywords={self}, % Additional keywords
}

\begin{document}
	
% Title page
\section{Beveztés}
\title[]{Adatbányászat a Gyakorlatban}
\subtitle{Tantárgyi útmutató}
\author[Kuknyó Dániel]{Kuknyó Dániel\\Budapesti Gazdasági Egyetem}
\date{2024/25\\1.félév}
\makebeamertitle

\begin{frame}{Tartalom}
\tableofcontents{}
\end{frame}

\begin{frame}{Elérhetőségek}
\begin{center}
\begin{itemize}
	\item \textbf{E-mail}: \href{mailto:daniel.kuknyo@mailbox.org}{daniel.kuknyo@mailbox.org}
	\item \textbf{Messenger}: \href{https://www.facebook.com/dani.kkny/}{Dani Kuknyo}
	\item \textbf{Coospace üzenet}: Y80L35
	\item \textbf{A tárgy Git tárhelye}: \href{https://github.com/basictask/Adatbanyaszat}{basictask/Adatbanyaszat}
	\item \textbf{Teams-en nem vagyok rendszeresen elérhető.} 
\end{itemize}
\end{center}
A fenti címeken lehetősége van minden hallgatónak kérdezésre és konzultációt egyeztetni. A konzultáció platformja Teams, előre megbeszélt időpontban.
\end{frame}

\begin{frame}{A félév tematikája}

\begin{itemize}
	\item[-] Bevezetés
	\begin{itemize}
		\item[\textbf{1.}] \textbf{Verziókezelés és Git}
	\end{itemize}
	\item[-] Dash keretrendszer
	\begin{itemize}
		\item[\textbf{2.}] \textbf{Bevezetés a Dash keretrendszerbe}
		\item[\textbf{3.}] \textbf{Diagramok létrehozása Dash alatt}
		\item[\textbf{4.}] \textbf{Pontszórási diagramok, interaktív térképek}
		\item[\textbf{5.}] \textbf{Gyakorisági adatok, dinamikus komponensek, gépi tanulás}
		\item[\textbf{6.}] \textbf{Felhasználói komponensek, többlapos műszerfalak}
	\end{itemize}
	\item[-] Mélytanulás
	\begin{itemize}
		\item[\textbf{7.}] \textbf{Bevezetés a mesterséges mélytanulásba}
		\item[\textbf{8.}] \textbf{Objektum detekció}
		\item[\textbf{9.}] \textbf{Egyed szegmentáció}
		\item[\textbf{10.}] \textbf{Visszacsatolásos neurális hálózatok}
		\item[\textbf{11.}] \textbf{Transzformáló architektúrák}
	\end{itemize}
\end{itemize}
	
\end{frame}

\section{Követelmények}

\begin{frame}{}
	\tableofcontents[currentsection]
\end{frame}

\begin{frame}{Követelmények}
\begin{itemize}
	\item A félév során \textbf{2 gyakorlati beadandót} kell teljesíteni, egyet megerősítéses tanulás és egyet mesterséges mélytanulás témaköréből. 
	\item A beadandók \textbf{egyéniek, és hallgatónként más algoritmusokat kell implementálni}.
	\item Egyenként 50-50 pont elérhető. Ez \textbf{összesen 100} gyakorlati pont. 
	\item \textbf{Az egyedi munka elvárt és ellenőrzött}. Plágium esetén a munka érvénytelen. 
\end{itemize}
\par\smallskip
	\begin{columns}
	\begin{column}{.3\textwidth}
		\end{column}
			\begin{column}{.3\textwidth}
				\begin{block}{\begin{center}Ponthatárok\end{center}}
					\begin{center}
						$90 \leq p < 100 \Rightarrow 5$\\
						$80 \leq p < 90 \Rightarrow 4$\\
						$70 \leq p < 80 \Rightarrow 3$\\
						$60 \leq p < 70 \Rightarrow 2$\\
						$p < 60 \Rightarrow 1$
					\end{center}
				\end{block}
			\end{column}
		\begin{column}{.3\textwidth}
		\end{column}
	\end{columns}
\end{frame}

\begin{frame}{Beadás menete}
	\begin{itemize}
		\item A félév során mindenkinek létre kell hoznia egy \textbf{saját Git tárhelyet}, ahol a féléves munkáját fogja rögzíteni:
		\begin{itemize}
			\item A Git felhasználónév teljesen mindegy, de a név mezőbe a teljes nevet írjátok be. 
			\item A tárhely legyen privát. 
			\item Vegyetek fel engem, mint hozzájáruló fejlesztőt a tárhelyre \emph{basictask} felhasználónévvel. 
		\end{itemize}
		\item A munkák beadása Coospace felületen történik. \textbf{Csak egy linket várok, ami a beadandó feladathoz tartozó Git tárhelyre mutat}. Fájlokat és egyéb állományokat nem lehet feltölteni Coospace-re.
		\item \textbf{Késői beadásra nincs lehetőség}. Ha a határidő után történik mentés a tárhelyre, nem lesz figyelembe véve.
	\end{itemize}
\end{frame}

\section{Órai környezet telepítése}

\begin{frame}{}
	\tableofcontents[currentsection]
\end{frame}

\begin{frame}[fragile]{Órai anyagok letöltése}
	\begin{enumerate}
		\item Anaconda környezet telepítése \href{https://www.anaconda.com/download}{innen}
		\item Git telepítése \href{https://git-scm.com/downloads}{innen}
		\item Órai tárhely klónozása a számítógépre (Git bash):
		\begin{lstlisting}[language=bash]
git clone https://github.com/basictask/Adatbanyaszat.git
		\end{lstlisting}
		\item Új \texttt{conda} környezet létrehozása és aktiválása (Anaconda prompt):
		\begin{lstlisting}[language=bash]
conda create -n dash python=3.11
conda activate dash
		\end{lstlisting}
		\item A pro jekt gyökérmappájában állva a következő paranccsal lehet minden könyvtárat telepíteni (Anaconda prompt):
		\begin{lstlisting}[language=bash]
pip install -r requirements.txt
		\end{lstlisting}
	\end{enumerate}
\end{frame}

\begin{frame}{Javasolt fejlesztői környezetek}
	    \begin{itemize}
		\item \textbf{Dash alkalmazásokhoz}
		\begin{itemize}
			\item Pycharm Community/Professional (Egyetemi Hallgatóknak ingyenes a Professional)
			\item Spyder (Anaconda fejlesztői csomaggal elérhető)
		\end{itemize}
		\item \textbf{Jupyter notebook fájlokhoz}
		\begin{itemize}
			\item Jupyter notebook (Anaconda fejlesztői csomaggal ingyenes)
			\item Visual Studio Code (Jupyter bővítménnyel)
			\item Pycharm Professional Edition (Egyetemi Hallgatóknak ingyenes)
		\end{itemize}
	\end{itemize}
	\par\medskip
	A felsoroltakon kívül bármely más fejlesztői környezet is használható a kurzus alatt.
\end{frame}




\end{document}