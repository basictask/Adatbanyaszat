\documentclass{article}
\usepackage{listings}
\usepackage{hyperref}
\usepackage{titlesec}
\usepackage{titling}
\usepackage{xcolor}
\usepackage{float}
\usepackage{color}

\definecolor{lightgrey}{gray}{0.95}

% Definitions for program code sections
\lstset{
    language=bash,
    basicstyle=\ttfamily\footnotesize, % Monospace font
    backgroundcolor=\color{lightgrey}, % Background color
    frame=single, % Frame around the code
    keywordstyle=\color{black}, % Keywords color
    commentstyle=\color{black}, % Comments color
    stringstyle=\color{red}, % Strings color
    showstringspaces=false, % Do not show spaces in strings
    breaklines=true, % Automatically break long lines
}

\title{Adatbányászat a Gyakorlatban}
\pretitle{\begin{center}\LARGE\bfseries}
\posttitle{\end{center}}
\preauthor{\begin{center}\large}
\postauthor{\end{center}}
\predate{\begin{center}\large}
\postdate{\end{center}}
\author{Kuknyó Dániel}
\date{2024/25/1}

\begin{document}

\maketitle

\begin{center}
\large\textit{1. Gyakorlat: Verziókezelés}
\end{center}

\section{GitHub}
\subsection*{Regisztráció a GitHub-ra}
\href{www.github.com}{www.github.com}

\subsection*{Git telepítése}
\subsubsection*{Windows}
Git Bash (parancssoros Git) letöltése: \href{https://git-scm.com/downloads}{https://git-scm.com/downloads}\\
Asztali kliens: \href{https://desktop.github.com/}{https://desktop.github.com/}

\subsubsection*{Linux}
\begin{lstlisting}
(type -p wget >/dev/null || (sudo apt update && sudo apt-get install wget -y)) \
&& sudo mkdir -p -m 755 /etc/apt/keyrings \
&& wget -qO- https://cli.github.com/packages/githubcli-archive-keyring.gpg | sudo tee /etc/apt/keyrings/githubcli-archive-keyring.gpg > /dev/null \
&& sudo chmod go+r /etc/apt/keyrings/githubcli-archive-keyring.gpg \
&& echo "deb [arch=$(dpkg --print-architecture) signed-by=/etc/apt/keyrings/githubcli-archive-keyring.gpg] https://cli.github.com/packages stable main" | sudo tee /etc/apt/sources.list.d/github-cli.list > /dev/null \
&& sudo apt update \
&& sudo apt install gh -y
\end{lstlisting}
Ezután telepítés:
\begin{lstlisting}
sudo apt update
sudo apt install gh
\end{lstlisting}

\subsection*{Konfigurálás}
\subsubsection*{Parancssorban}

Név hozzáadása:
\begin{lstlisting}
git config --global user.name "Gipsz Jakab"
\end{lstlisting}

E-mail cím hozzáadása
\begin{lstlisting}
git config --global user.email "gipsz.jakab@gmail.com"
\end{lstlisting}

\subsubsection*{Asztali kliensen}

\texttt{File} / \texttt{Options} / \texttt{Accounts} / \texttt{Sign in} / Itt hitelesíteni kell a Githubot a böngészővel

\section{Git alapok}
\subsection*{Új tárhely létrehozása}
\subsubsection*{Parancssorban}
\begin{lstlisting}
$ git init
$ git add *.c
$ git add LICENSE
$ git commit -m 'Initial project version'
\end{lstlisting}

\subsubsection*{Asztali kliensen}
\texttt{File} / \texttt{New repository}
\begin{enumerate}
\item \textbf{Name}: A tárhely neve
\item \textbf{Description}: A tárhely rövid leírása
\item \textbf{Local path}: A tárhely mappája a helyi számítógépen
\item \textbf{Git ignore}: A gitignore-hoz adott fájlok nem lesznek feltöltve a tárhelyre. Itt érdemes kiválasztani az adott programnyelvhez tartozó ignore konfigurációt. A .gitignore a tárhely gyökerében lévő szöveges állomány, amihez saját magunk is hozzáírhatunk ha vannak olyan fájlok, amelyeket nem szeretnénk feltölteni a tárhelyre. Ilyenek lehetnek a nagyméretű állományok vagy személyes információt / konfigurációt tartalmazó állományok.
\item \textbf{License}: Milyen másolhatósági / továbbadhatósági szabályok vonatkoznak a tárhelyre. Ezt legtöbb esetben elhagyhatjuk.
\end{enumerate}

\subsection*{Gitignore szerkesztése}
Belépés a tárhely gyökerébe:
\begin{lstlisting}
cd c/Users/Jakab/GitHub/desktop-tutorial
\end{lstlisting}
Új ignore fájl létrehozása:
\begin{lstlisting}
nano .gitignore
\end{lstlisting}
Ezután minden sorban egy ignorált fált lehet megadni, vagy a * karakterrel egy általános definíciót megadni:
\begin{lstlisting}
videos/big_video.mp4
passwords/*
*.zip
*.rar
*.jpg
\end{lstlisting}
Mentés és kilépés a szerkesztőből:
\begin{lstlisting}
ctrl + o
ctrl + x
\end{lstlisting}
Ezek a definíciók példa jellegűek. A valósan ignorált fájlok és típusok tárhelytől függően változhatnak.

\subsection*{Létező tárhely klónozása}
Egy tárolót a \texttt{git clone <url>} paranccsal lehetséges klónozni. Például, a libgit2 nevű Git linkelhető könyvtár klónozásának parancsa:
\begin{lstlisting}
git clone https://github.com/libgit2/libgit2
\end{lstlisting}

\section{Fejlesztési ágak kezelése}

\subsection*{Fájlok státuszának ellenőrzése}
A fő eszköz, amellyel meg lehet határozni, hogy mely fájlok milyen állapotban vannak, a \texttt{git status} parancs. Ha közvetlenül egy klónozás után fut ez a parancs, az output hasonló lesz:
\begin{lstlisting}
$ git status
On branch master
Your branch is up-to-date with 'origin/master'.
nothing to commit, working tree clean
\end{lstlisting}
\begin{minipage}{\linewidth}
Ha például egy README fájl hozzáadása futtatódik a parancs, a kimenet:
\begin{lstlisting}
$ echo 'My Project' > README
$ git status
On branch master
Your branch is up-to-date with 'origin/master'.
Untracked files:
(use "git add <file>..." to include in what will be committed)
README
nothing added to commit but untracked files present (use "git add" to track)
\end{lstlisting}
\end{minipage}

\subsection*{Fájlok indexelése}
\subsubsection*{Új fájlok esetén}
Amikor egy új fájl létrejön a munkakönyvtárban, az alapértelmezés szerint nem kerül be a Git verziókezelésébe. A \texttt{git add <filename>} parancs segítségével hozzáadható ez az új fájl az indexhez. Ez azt jelenti, hogy a fájl készen áll arra, hogy a következő commit során bekerüljön a verziókezelésbe.
\begin{lstlisting}
$ git add README
$ git status
On branch master
Your branch is up-to-date with 'origin/master'.
Changes to be committed:
(use "git restore --staged <file>..." to unstage)
new file:
 README
\end{lstlisting}

\subsubsection*{Létező, módosított fájlok esetén}
Amikor egy már verziókezelés alatt álló fájl módosításra kerül a munkakönyvtárban, a változtatások alapértelmezés szerint nem kerülnek azonnal az indexbe. A \texttt{git add <filename>} parancs segítségével a módosított fájl hozzáadható az indexhez. Ez azt jelenti, hogy a fájlban történt változtatások készen állnak arra, hogy a következő commit során bekerüljenek a verziókezelésbe.
\begin{lstlisting}
$ git add CONTRIBUTING.md
$ git status
On branch master
Your branch is up-to-date with 'origin/master'.
Changes to be committed:
(use "git reset HEAD <file>..." to unstage)
new file:
 README
modified:
 CONTRIBUTING.md
\end{lstlisting}

\subsubsection*{Fájlok eltávolítása az indexből}
Egy fájl Gitből való eltávolításához a követett fájlok közül kell eltávolítani, majd commitolni kell a változtatásokat. Ezt a \texttt{git rm} parancs képes elvégezni. Ez a munkamappából is eltávolítja a fájlt.
\begin{lstlisting}
$ git rm PROJECTS.md
rm 'PROJECTS.md'
$ git status
On branch master
Your branch is up-to-date with 'origin/master'.
Changes to be committed:
(use "git reset HEAD <file>..." to unstage)
deleted:
 PROJECTS.md
\end{lstlisting}

\subsubsection*{Egy indexelt fájl törlése az indexből}
Ez a parancs nem fogja eltávolítani a fájlt a munkamappából, csak a Git követést fogja leállítani. Ez a változtatás commit és push esetén a távoli tárhelyről törölni fogja a fájlt.
\begin{lstlisting}
$ git reset HEAD CONTRIBUTING.md
Unstaged changes after reset:
M
 CONTRIBUTING.md
$ git status
On branch master
Changes not staged for commit:
(use "git add <file>..." to update what will be committed)
(use "git checkout -- <file>..." to discard changes in working directory)
modified:
 CONTRIBUTING.md
\end{lstlisting}

\subsection*{Változtatások commitolása}
Commit során a változtatások elmentődnek egy lokális pillanatképbe. A pillanatképet a távoli tárhelyre \texttt{push} során lehetséges feltölteni.
\begin{lstlisting}
$ git commit -m "Story 182: fix benchmarks for speed"
[master 463dc4f] Story 182: fix benchmarks for speed
2 files changed, 2 insertions(+)
create mode 100644 README
\end{lstlisting}
A \texttt{-m} paraméter kötelező, és a commit üzenetet (message) adja meg.

\section{Távoli tárhelyek}
\subsection*{Változtatások lekérdezése}
A \texttt{fetch} parancs lekérdezi a változtatott fájlok listáját. Ezután letöltődnek azok az ágak és commitok, amelyek még nincsenek meg a lokális tárhelyen.
\begin{lstlisting}
$ git fetch origin
\end{lstlisting}

\subsection*{Változtatások befésülése}
A \texttt{fetch} parancs által letöltött változtatások befésülését a \texttt{git pull} parancs képes elvégezni. Ez a lokális fájlrenszerben frissíti a fájlokat a távolira:
\begin{lstlisting}
git pull origin main
\end{lstlisting}
Ebben a példában az \texttt{origin} távoli tároló \texttt{main} fejlesztési ágáról történik a letöltés.

\subsection*{Új fejlesztési ág létrehozása}
A \texttt{git branch <branchname>} parancs lokálisan létrehoz egy új fejlesztési ágat. Egyenértékű vele a \texttt{checkout} parancs a \texttt{-b} argumentummal.
\begin{lstlisting}
$ git checkout testing
\end{lstlisting}
vagy
\begin{lstlisting}
$ git checkout -b testing
\end{lstlisting}
Az aktuálisan megjelölt ágat a \texttt{HEAD} mutató tartja számon. Ezt bármelyik pillanatban le lehet kérdezni a \texttt{git log} segítségével:
\begin{lstlisting}
$ git log --oneline --decorate
f30ab (HEAD -> master, testing) Add feature #32 - ability to add new formats to the
central interface
34ac2 Fix bug #1328 - stack overflow under certain conditions
98ca9 Initial commit
\end{lstlisting}

\subsection*{Váltás ágak között}
A váltás a \texttt{git checkout} paranccsal történik:
\begin{lstlisting}
$ git checkout testing
\end{lstlisting}
Ez a parancs a \texttt{HEAD} mutatót átmozgatja a \texttt{testing} ágra.
\vspace{0.3cm}
A Git ágak közötti váltásra szabályok érvényesek, nem lehet minden esetben megtenni, és némelyik esetben pedig összeolvasztási konfliktusokhoz vezethet.
\begin{itemize}
    \item \textbf{Tiszta munkakönyvtár}: Ágak közötti váltás csak akkor lehetséges, ha a munkakönyvtár tiszta, azaz nincsenek nem commitolt változtatások. Ha vannak módosítások, azokat commitolni vagy stashelni kell a váltás előtt.

    \item \textbf{Stash használata}: Ha a változtatások nem kívánatosak a jelenlegi ágban, de nem szeretnének commitolni, a git stash parancs használható. Ez ideiglenesen elmenti a változtatásokat, így lehetővé teszi az ágak közötti váltást.

    \item \textbf{Konfliktusok elkerülése}: Ha egy másik ágra történő váltás során olyan fájlok vannak módosítva, amelyek a célágban is módosultak, merge konfliktusok léphetnek fel. Ezeket a konfliktusokat manuálisan kell megoldani a váltás után.

    \item \textbf{Követetlen fájlok}: Az ágak közötti váltás nem érinti az untracked státuszban lévő (nem verziókezelt) fájlokat. Ezek a fájlok megmaradnak a munkakönyvtárban.
\end{itemize}

\subsection*{Stash}
A \texttt{git stash} parancs ideiglenesen elmenti a munkakönyvtárban lévő módosításokat anélkül, hogy commitolni kellene őket. Ez hasznos lehet, ha váltani kell egy másik ágra, de a jelenlegi módosításokat még nem szeretnénk commitolni. A \texttt{git stash} parancs elmenti a változtatásokat egy "stash" nevű területre, ahonnan később vissza lehet őket állítani.
\vspace{0.2cm}
Módosítások stashelése:
\begin{lstlisting}
$ git stash -m "Temporary changes"
\end{lstlisting}
Ezután biztonságosal lehet váltani a \texttt{main} ágra:
\begin{lstlisting}
$ git checkout main
\end{lstlisting}
A stashelt változtatások megtekintése:
\begin{lstlisting}
$ git stash list
stash@{0}: On feature-branch: Temporary changes
\end{lstlisting}
Változtatások kivétele a stashből:
\begin{lstlisting}
$ git stash pop stash@{0}
\end{lstlisting}

\subsection*{Összefésülések}
Ágak összefésülése a \texttt{git merge} paranccsal lehetséges. Ebben az esetben a fejlesztési előzmények a két ág esetén eltérnek. Mivel a commit ág nem közvetlen őse a célágnak, a Git egy háromágú összefésülést végez a két pillanatkép és a közös ős segítségével.
\begin{lstlisting}
$ git checkout master
Switched to branch 'master'
$ git merge iss53
Merge made by the 'recursive' strategy.
index.html |
 1 +
1 file changed, 1 insertion(+)
\end{lstlisting}

\end{document}
