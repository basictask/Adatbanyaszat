% Settings for the default beamer theme
\documentclass[english, aspectratio=169]{beamer}
\usepackage[T1]{fontenc}
\usepackage[utf8]{inputenc}
\usepackage{adjustbox}
\usepackage{tabularx}
\usepackage{listings}
\usepackage{graphicx}
\usepackage{array}
\usepackage{babel}
\usepackage[ruled,vlined]{algorithm2e}
\usepackage{blkarray}
\SetAlgorithmName{Algoritmus}{algoritmus}{List of Algorithms}
\setcounter{secnumdepth}{3}
\setcounter{tocdepth}{3}


\makeatletter

\newcommand\makebeamertitle{\frame{\maketitle}}

% (ERT) argument for the TOC
\AtBeginDocument{%
  \let\origtableofcontents=\tableofcontents
  \def\tableofcontents{\@ifnextchar[{\origtableofcontents}{\gobbletableofcontents}}
  \def\gobbletableofcontents#1{\origtableofcontents}
}

% Theme settings
\usetheme{Frankfurt}
\usecolortheme{default}
\usefonttheme[onlymath]{serif}

% Template settings
\setbeamertemplate{navigation symbols}{}
\setbeamertemplate{blocks}[rounded][shadow=false]
\setbeamertemplate{title page}[default][colsep=-4bp, rounded=true, shadow=false]
\makeatother

% Custom color definitions
\definecolor{lightgrey}{gray}{0.95}
\definecolor{DarkerGreen}{RGB}{0,85,0} % Adjust the RGB values as needed

% Use the newly defined color in Beamer theme elements
\setbeamercolor{structure}{fg=DarkerGreen} % Changes basic structural elements to Darker Green
\setbeamercolor{title in head/foot}{bg=DarkerGreen} % Changes the title in header/footer to Darker Green

% Definitions for program code sections
\lstset{
	language=bash,
	basicstyle=\ttfamily\footnotesize, % Monospace font
	backgroundcolor=\color{lightgrey}, % Background color
	frame=single, % Frame around the code
	keywordstyle=\color{black}, % Keywords color
	commentstyle=\color{black}, % Comments color
	stringstyle=\color{red}, % Strings color
	showstringspaces=false, % Do not show spaces in strings
	breaklines=true, % Automatically break long lines
}

\lstset{
	language=python,
	basicstyle=\ttfamily\scriptsize, % Basic font style
	keywordstyle=\bfseries\color{blue}, % Keywords in bold and blue
	stringstyle=\color{red}, % Strings in red
	commentstyle=\color{green!50!black}, % Comments in green
	showstringspaces=false, % Do not show spaces in strings
	numbers=left, % Line numbers on the left
	numberstyle=\tiny\color{gray}, % Line number style
	stepnumber=1, % Line number step
	numbersep=5pt, % Distance of line numbers from code
	frame=single, % Frame around the code
	rulecolor=\color{black}, % Frame color
	tabsize=2, % Tab size
	breaklines=true, % Automatic line breaking
	breakatwhitespace=false, % Break lines at whitespace
	captionpos=b, % Caption position
	escapeinside={\%*}{*)}, % Escape to LaTeX
	morekeywords={self}, % Additional keywords
	literate={á}{{\'a}}1
	{é}{{\'e}}1
	{í}{{\'i}}1
	{ó}{{\'o}}1
	{ú}{{\'u}}1
	{ő}{{\H{o}}}1
	{ű}{{\H{u}}}1
	{Á}{{\'A}}1
	{É}{{\'E}}1
	{Í}{{\'I}}1	
	{Ó}{{\'O}}1	
	{Ú}{{\'U}}1
	{Ő}{{\H{O}}}1
	{Ű}{{\H{U}}}1
	{Ö}{{\"O}}1
	{Ü}{{\"U}}1
	{ö}{{\"o}}1
	{ü}{{\"u}}1
}


\begin{document}

% Title page
\section{Bevezetés}
\title[]{Adatbányászat a Gyakorlatban}
\subtitle{4. Gyakorlat: Pontdiagramok és viselkedésük}
\author[Kuknyó Dániel]{Kuknyó Dániel\\Budapesti Gazdasági Egyetem}
\date{2024/25\\1.félév}
\makebeamertitle

% Table of contents slide
\begin{frame}
\tableofcontents{}
\end{frame}

% Table of contents of the current section
\begin{frame}
\tableofcontents[currentsection]
\end{frame}

\begin{frame}{Alapvető pontdiagramok}
	\begin{columns}
		\begin{column}{.5\textwidth}
			A plotly könyvtárban a pontdiagramokat a \texttt{graph\_objects} valósítja meg. Az alapvető jelölőtípusok:
			\begin{itemize}
				\item \texttt{markers}: Csak jelölők
				\item \texttt{lines}: Csak vonalak
				\item \texttt{text}: Csak szöveget
				\item \texttt{markers+lines}: Jelölők és vonalak
				\item \texttt{markers+text}: Jelölők és szöveg
				\item \texttt{lines+text}: Vonalak és szöveg
				\item \texttt{markers+lines+text}: Jelölők, vonalak és szöveg
			\end{itemize}
		\end{column}
		\begin{column}{.5\textwidth}
			\begin{center}
				\includegraphics<1>[width=7cm, height=7cm, keepaspectratio]{images/scatter_1.png}
				\includegraphics<2>[width=7cm, height=7cm, keepaspectratio]{images/scatter_2.png}
				\includegraphics<3>[width=7cm, height=7cm, keepaspectratio]{images/scatter_3.png}
				\includegraphics<4>[width=7cm, height=7cm, keepaspectratio]{images/scatter_4.png}
				\includegraphics<5>[width=7cm, height=7cm, keepaspectratio]{images/scatter_5.png}
				\includegraphics<6>[width=7cm, height=7cm, keepaspectratio]{images/scatter_6.png}
				\includegraphics<7>[width=7cm, height=7cm, keepaspectratio]{images/scatter_7.png}
			\end{center}
		\end{column}
	\end{columns}
\end{frame}

\begin{frame}[fragile]{Több pontszórási nyomvonal egy diagramon}
	\begin{columns}
		\begin{column}{.5\textwidth}
			Ahhoz, hogy több vonal jelenjen meg egy vásznon belül, egyesével kell hozzáadni a diagramhoz a megfelelő adatokkal definiált \texttt{trace} objektumokat.
			\begin{lstlisting}[language=python]
for country in countries:
	df_country = df[df['Country Name'] == country]
	fig.add_scatter(x=df_country['year'], y=df_country[perc_pov_19], name=country, mode='markers+lines')
			\end{lstlisting}
		\end{column}
		\begin{column}{.5\textwidth}
			
		\end{column}
	\end{columns}
\end{frame}

\end{document}











