% Settings for the default beamer theme
\documentclass[english, aspectratio=169]{beamer}
\usepackage[T1]{fontenc}
\usepackage[utf8]{inputenc}
\usepackage{listings}
\usepackage{tabularx}
\usepackage{graphicx}
\usepackage{babel}
\usepackage[ruled,vlined]{algorithm2e}
\SetAlgorithmName{Algoritmus}{algoritmus}{List of Algorithms}
\setcounter{secnumdepth}{3}
\setcounter{tocdepth}{3}

\makeatletter

\newcommand\makebeamertitle{\frame{\maketitle}}

% (ERT) argument for the TOC
\AtBeginDocument{%
  \let\origtableofcontents=\tableofcontents
  \def\tableofcontents{\@ifnextchar[{\origtableofcontents}{\gobbletableofcontents}}
  \def\gobbletableofcontents#1{\origtableofcontents}
}

% Theme settings
\usetheme{Frankfurt}
\usecolortheme{default}
\usefonttheme[onlymath]{serif}

% Template settings
\setbeamertemplate{navigation symbols}{}
\setbeamertemplate{blocks}[rounded][shadow=false]
\setbeamertemplate{title page}[default][colsep=-4bp, rounded=true, shadow=false]
\makeatother

% Define a custom darker red color
\definecolor{DarkerGreen}{RGB}{0,85,0} % Adjust the RGB values as needed
\definecolor{lightgrey}{gray}{0.95}

% Use the newly defined color in Beamer theme elements
\setbeamercolor{structure}{fg=DarkerGreen} % Changes basic structural elements to Darker Green
\setbeamercolor{title in head/foot}{bg=DarkerGreen} % Changes the title in header/footer to Darker Green

% Definitions for program code sections
\lstset{
    language=bash,
    basicstyle=\ttfamily\footnotesize, % Monospace font
    backgroundcolor=\color{lightgrey}, % Background color
    frame=single, % Frame around the code
    keywordstyle=\color{black}, % Keywords color
    commentstyle=\color{black}, % Comments color
    stringstyle=\color{red}, % Strings color
    showstringspaces=false, % Do not show spaces in strings
    breaklines=true, % Automatically break long lines
}

\begin{document}

% Title page
\section{Bevezetés}
\title[]{Adatbányászat a Gyakorlatban}
\subtitle{2. Gyakorlat: Dash Alapok}
\author[Kuknyó Dániel]{Kuknyó Dániel\\Budapesti Gazdasági Egyetem}
\date{2024/25\\1.félév}
\makebeamertitle

% Table of contents slide
\begin{frame}
\tableofcontents{}
\end{frame}

% Table of contents of the current section
\begin{frame}
\tableofcontents[currentsection]
\end{frame}

\begin{frame}[fragile]{Órai környezet telepítése}
\begin{enumerate}
  \item Új python környezet létrehozása \texttt{dash} (vagy bármilyen más tetszőleges) néven:
  \begin{lstlisting}
  $ conda create --name dash python=3.12
  \end{lstlisting}
  \item Környezet aktiválása:
  \begin{lstlisting}
  $ conda activate dash
  \end{lstlisting}
  \item Az órai tárhely klónozása a Git tárhelyről:
  \begin{lstlisting}
  $ git clone https://github.com/basictask/Adatbanyaszat.git
  \end{lstlisting}
  \item Belépés a klónozott könyvtárba és a környezeti változók telepítése:
  \begin{lstlisting}
  $ cd Adatbanyaszat
  $ pip install -r requirements.txt
  \end{lstlisting}
\end{enumerate}
\end{frame}

\begin{frame}[fragile]{Mi az a Dash?}
A Dash python könyvtár lehetővé teszi interaktív adatalapú alkalmazások létrehozását színtisztán Python nyelv segítségével. A Dash keretrendszer a Flask könyvtárat használja backend műveletekre, diagramokat a Plotly könyvtárral hoz létre, és a teljes alkalmazást egyetlen React alkalmazásként rendereli le.
\begin{center}
\includegraphics[width=3cm, height=3cm, keepaspectratio]{images/dash_1.png}
\includegraphics[width=3cm, height=3cm, keepaspectratio]{images/dash_2.png}
\includegraphics[width=3cm, height=3cm, keepaspectratio]{images/dash_3.png}
\includegraphics[width=3cm, height=3cm, keepaspectratio]{images/dash_4.png}
\end{center}
\end{frame}

\end{document}
